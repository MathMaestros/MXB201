%!TEX program = xelatex
\documentclass{article}
\usepackage{LaTeX-Submodule/template}

% Additional packages & macros

% Header and footer
\newcommand{\unitName}{Advanced Linear Algebra}
\newcommand{\unitTime}{Semester 1, 2022}
\newcommand{\unitCoordinator}{Prof Timothy Moroney}
\newcommand{\documentAuthors}{\textsc{Tarang Janawalkar}}

\fancyhead[L]{\unitName}
\fancyhead[R]{\leftmark}
\fancyfoot[C]{\thepage}

% Copyright
\usepackage[
    type={CC},
    modifier={by-nc-sa},
    version={4.0},
    imagewidth={5em},
    hyphenation={raggedright}
]{doclicense}

\date{}

\begin{document}
%
\begin{titlepage}
    \vspace*{\fill}
    \begin{center}
        \LARGE{\textbf{\unitName}} \\[0.1in]
        \normalsize{\unitTime} \\[0.2in]
        \normalsize\textit{\unitCoordinator} \\[0.2in]
        \documentAuthors
    \end{center}
    \vspace*{\fill}
    \doclicenseThis
    \thispagestyle{empty}
\end{titlepage}
\newpage
%
\tableofcontents
\newpage
%
\section{Fundamental Concepts of Linear Algebra}
\subsection{Row Echelon Form}
As studied in Linear Algebra, we can solve linear systems by
applying the following elementary row operations to any matrix \(\symbf{A}\).
\begin{enumerate}[label=Type \Roman*.]
    \item Exchange any two rows.
    \item Multiply any row by a constant.
    \item Add a multiple of one row to another row.
\end{enumerate}
This allows us to reduce \(\symbf{A}\) into \textbf{row echelon form}
such that the entries below the main diagonal are zero:
\begin{equation*}
    \symbf{R}_{\mathrm{ref}} =
    \begin{bmatrix}
        r_{11} & r_{12} & \cdots & r_{1n} \\
               & r_{22} & \cdots & r_{2n} \\
               &        & \ddots & \vdots \\
               &        &        & r_{mn}
    \end{bmatrix}
\end{equation*}
\subsection{Elementary Matrix}
Mathematically, we can represent these row operations as a matrix
which is left multiplied to \(\symbf{A}\).
\begin{definition}[Elementary matrix]
    An elementary matrix \(\symbf{E}_i\) is constructed by applying a row operation to the elementary matrix \(\symbf{I}_m\).
    Consider a 3 by 4 matrix \(\symbf{A}\); a common first elementary row operation might be
    \begin{equation*}
        r_2 \leftarrow r_2 - \frac{a_{21}}{a_{11}} r_1
    \end{equation*}
    which when applied to \(\symbf{I}_3\) yields
    \begin{equation*}
        \symbf{E}_1 =
        \begin{bmatrix}
            1                      & 0 & 0 \\
            -\frac{a_{21}}{a_{11}} & 1 & 0 \\
            0                      & 0 & 1
        \end{bmatrix}
    \end{equation*}
    where the 1 subscript simply indicates the first of many elementary row operations.
    Left multiplying this to an arbitrary \(\symbf{A}\) gives
    \begin{align*}
        \symbf{E}_1 \symbf{A} & = 
        \begin{bmatrix}
            1                      & 0 & 0 \\
            -\frac{a_{21}}{a_{11}} & 1 & 0 \\
            0                      & 0 & 1
        \end{bmatrix}
        \begin{bmatrix}
            a_{11} & a_{12} & a_{13} & a_{14} \\
            a_{21} & a_{22} & a_{23} & a_{24} \\
            a_{31} & a_{32} & a_{33} & a_{34}
        \end{bmatrix} \\
        & =
        \begin{bmatrix}
            a_{11} & a_{12}                              & a_{13}                              & a_{14}                              \\
            0      & a_{22}-\frac{a_{12} a_{21}}{a_{11}} & a_{23}-\frac{a_{13} a_{21}}{a_{11}} & a_{24}-\frac{a_{14} a_{21}}{a_{11}} \\
            a_{31} & a_{32}                              & a_{33}                              & a_{34}
        \end{bmatrix}
    \end{align*}
    which has the desired result of eliminating the first column of the second row.
\end{definition}
\subsection{Reduced Row Echelon Form}
As there are infinitely many ways to reduce a matrix to row echelon form, we typically
reduce \(\symbf{R}_{\mathrm{ref}}\) further into \textbf{reduced row echelon form} which is
a unique reduction for every \(\symbf{A}\).

This matrix \(\symbf{R}_{\mathrm{rref}}\) (or simply \(\symbf{R}\)) generally requires \(m \times n\)
elementary row operations and is only useful for theoretical analysis.
In reduced row echelon form, any entries in the same column as a pivot must be 0, and each pivot is 1.
\subsection{Elimination Matrix}
The elementary matrices involved in row reduction can be expressed as a single matrix containing every
each row operation.
\begin{align*}
   \symbf{E}_9 \symbf{E}_8 \dots \symbf{E}_2 \symbf{E}_1 \symbf{A} & = \symbf{E} \symbf{A} \\
                                                                   & = \symbf{R} 
\end{align*}
\subsection{Linear Systems}
Given the linear system \(\symbf{A} \symbfit{x} = \symbfit{b}\)
we can augment \(\symbf{A}\) with \(\symbfit{b}\) to draw conclusions about the solutions.

If we left multiply the elimination matrix \(\symbf{E}\) to 
\(\begin{bmatrix}[c|c]
    \symbf{A} & \symbfit{b}    
\end{bmatrix}\)
we can apply the same operations to \(\symbfit{b}\).
\begin{align*}
    \symbf{E}
    \begin{bmatrix}[c|c]
        \symbf{A} & \symbfit{b}    
    \end{bmatrix} & = 
    \begin{bmatrix}[c|c]
        \symbf{E} \symbf{A} & \symbf{E} \symbfit{b}
    \end{bmatrix} \\
    & = \begin{bmatrix}[c|c]
        \symbf{R} & \symbfit{z}    
    \end{bmatrix}
\end{align*}
Therefore
\begin{equation*}
    \symbf{R} \symbfit{x} = \symbfit{z}
\end{equation*}
\subsection{Consistency of a Linear System}
After reducing the matrix \(\symbf{A}\) to \(\symbf{R}\), we can summarise 
certain characteristics about \(\symbf{A}\).
\subsubsection{Basic and Free Variables}
Identifying the pivots in \(\symbf{R}\) allows us to determine
the dimensions of various subspaces of \(\symbf{A}\).
\begin{definition}[Basic variables]
    The columns that a pivot corresponds to are known as basic variables (or leading variables).
\end{definition}
\begin{definition}[Free variables]
    Any columns not corresponding to any pivots are known as free variables (or parameters).
    Consequently, any variables that are not basic variables are free variables.
\end{definition}
In the following example, \(x_1\), \(x_3\), and \(x_4\) are basic variables, whereas \(x_2\) and \(x_5\) are free variables.
\begin{equation*}
    \begin{bmatrix}
        1 & 2 & 3 & 4 & 5 \\
        0 & 0 & 1 & 6 & 7 \\
        0 & 0 & 0 & 1 & 8
    \end{bmatrix}
\end{equation*}
When using backward substitution to solve \(\symbf{R} \symbfit{x} = \symbf{z}\), we assign new variables
to any free variables to indicate that they are parameters to the system.
\subsubsection{Singular Matrices}
After reducing \(\symbf{A}\) to \(\symbf{R}\) we can consider its 
\end{document}
